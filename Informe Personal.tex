\documentclass{article}

\usepackage[utf8]{inputenc} %para usar acentos.
\usepackage{hyperref}

\author{Daniel Peiró}
\title{Informe Personal}

\begin{document}

\maketitle

\section{Roles dentro del Equipo}
\begin{itemize}
\item Responsable del Documento de Especificación de Requisitos (SRS).
\item Integrante del Equipo de Programación.
\item Responsable del Equipo de Programación.
\end{itemize}


\section{Trabajo Asignado y Realizado}
Se divide entre los distintos roles.
\subsection{Responsable de SRS:}
\begin{itemize}
  \item Elaboración de SRS junto con todo el Equipo de Ingeniería
  (principalmente en una sesión de unas 3 horas.)
  \item Redacción del Glosario (sustantivos, los verbos los realizó Víctor,
  aunque los edité después).
  \item Revisión, edición y redacción del SRS una vez recibido el trabajo
  individual de cada uno de los componentes del equipo (la Arquitectura la realizó el equipo de Diseño
  como es lógico, el resto se repartió entre los demás), en conjunto con Jorge
  Camarero (PM) que se encargó de pasarlo a Latex.
  \item Presentación del estado del SRS en la reunión de revisión del proyecto
  del 12/3. (Elaborada en Powerpoint, Jorge Camarero (PM) la pasó a
  Latex/Beamer).
  \item Edición del SRS en Latex con el feedback recibido (todo salvo la
  arquitectura que fue revisada por el equipo de Diseño como es lógico).
\end{itemize}
\subsection{Integrante del Equipo de Programación:}
Todo el código que he escrito para el proyecto se puede encontrar en este
repositorio: \url{https://github.com/xpeiro/computadores-III}  (el
código en las carpetas include y remoteApi es de VREP):\\ \\
KheperaSimCMD\_src: Código de prototipo de caso de uso virtual. \\
KheperaSimGUI\_src: Código de caso de uso virtual + IU. \\
KheperaFisCMD: Código (muy simple) adaptado de caso de uso físico. \\
demo2khepera.ttt: Escena de VREP de demostración del programa. \\
kh3\_noplugin.ttt: Modelo VREP del Khepera que elimina el plugin predeterminado
y abre un puerto de API remota en 20001. \\
README: Instrucciones de prerequisitos, instalación, construcción y ejecución
de KheperaSimGUI (caso de uso virtual + IU).
	
\begin{itemize}
  \item Programación del caso de uso virtual y el caso de uso IU (ver SRS).
  Incluye:
	  \begin{itemize}
	    \item Estudio de API Remota de VREP.
	  	\item Estudio de los ejemplos de Cliente/Servidor en C++ ofrecidos por VREP
	  	(bubbleRobClient.cpp y bubbleRobServer.cpp en carpeta programming)
	  	\item Estudio de C++ (sintaxis básica y estructura de objetos básica).
	  	\item Programación de Prototipo por línea de comandos en C++ (permite
	  	ejecutar código de control aportando IP y Puerto y devuelve la posición
	  	instantánea). Creación del Makefile para facilitar el build (deprec. en
	  	siguientes versiones por build propio de Qt). Prueba en Windows/Linux del
	  	Prototipo.
	  	\item Diseño de estructura de clases I\_Control, Control y Demo que permite
	  	al usuario programar la clase Control implementando I\_Control (clase
	  	abstracta $ \approx $ interfaz) sin necesidad de preocuparse del resto del
	  	programa,  además de usar la clase pre-programada Demo que también
	  	implementa I\_Control (se incluye como ejemplo/utilidad ya que incluye interrupciones de giro y marcha atrás/adelante).
	  	\item Selección de Qt como API para la GUI. Estudio de Qt
	  	(básico) y QtCreator (IDE).
	  	\item Diseño y Programación de la GUI en Qt añadiendo todas las
	  	características que esta presenta: control de simulación (inicio
	  	/pausa/ reanudación/ parada), visualización de velocidad y posición en
	  	tiempo real con selección de robot a monitorizar, interrupciones
	  	predeterminadas (izq,der,adelante,atrás) con control de velocidad y selección de robot a interrumpir, interrupciones
	  	programadas por el usuario, detección de errores (no muy exhaustiva, solo lo
	  	esencial) y log de acciones (básico) en GUI y en archivo log.
	  	\item Redacción de README que explica la instalación, construcción y
	  	ejecución del programa tanto en Windows (VS2012+) como Linux(GCC).
	  	\item Prueba del programa en Windows/Linux intercambiando Hosts
	  	(Simulador/Controlador) para asegurar el funcionamiento en cualquier
	  	situación.
	  \end{itemize}
	\item Adaptación del código realizado por Carlos para el caso de uso físico.
	\subitem Empleaba una biblioteca multiplataforma llamada MRCore para implementar las
	comunicaciones entre un programa KheperaControl y otro KheperaSlave. Como el código era
	relativamente simple (Enviar una señal desde Control a Slave) y podía hacerse
	con un socket (técnicamente no multiplataforma), adapté el código para eliminar
	la necesidad de usar MRCore, sin perjuicio de que siguiera siendo
	multiplataforma (usa macros para definir includes y demás según la plataforma
	detectada por el compilador \_WIN32 ó \_\_linux\_\_). Esta adaptación la
	propuse como alternativa al uso de MRCore, ya que lo veía más simple, pero dejé
	a elección de Carlos continuar usando MRCore si le parecía mejor. En cualquier
	caso el código está lejos de ser una implementación completa del caso de uso
	físico. Control solo envía un char por un socket y Slave presenta un texto
	según lo recibido sea un char 'I' o 'P'. Habría que desarrollarlo para que al
	recibir la señal, Slave ejecutara kh3\_init() como	estipula el requisito R5.
\end{itemize}

\subsection{Responsable del Equipo de Programación}

\begin{itemize}
  \item Asignar el estudio de la API Remota de VREP a los tres integrantes del
  equipo, con el objetivo de encontrar posibles implementaciones (esto después
  de la clase del 12/3 en la que nos mostraste la existencia de la API Remota).
  \item Asignar el caso de uso físico a Carlos e Ignacio, y el caso de uso
  virtual e IU a mi (la asignación se hizo así porque el día 21/3 ya había
  realizado el prototipo del caso de uso virtual, mientras que el resto de
  integrantes aún no habían empezado a trabajar). Con esta
  organización, yo como responsable de programación me comprometía a integrar en
  la GUI el código del caso de uso físico que escribieran Carlos e Ignacio.
  \item Reuniones con el equipo de Diseño para consultar dudas sobre el
  funcionamiento del código, estructura etc. en lo que respecta a la
  arquitectura del sistema (Una con el equipo completo de Diseño y una con
  David).
  \item Consulta de dudas con el equipo de Test y Doc. (Marcial y Víctor) con
  respecto al README y algunos problemas/errores del programa.
\end{itemize}

\subsection{Misc.}
 \begin{itemize}
   \item Aprendizaje de Latex, usado para editar el SRS (creado por Jorge) y
   para crear este informe (nivel básico).
   \item Aprendizaje de Eclipse (básico). En principio se iba a usar para todo
   el proyecto pero al final, debido a la facilidad que QtCreator ofrece para
   editar código de GUI Qt solo se ha usado para editar Latex con el plugin
   \href{http://texlipse.sourceforge.net/}{Texlipse} y para hacer el prototipo,
   la adaptación del caso de uso físico y correcciones pequeñas de código, en
   conjunto con el editor de texto \href{http://www.sublimetext.com/}{Sublime Text}.
 \end{itemize}

\section{Trabajo Asignado y no Realizado / Temas Pendientes}
Trabajo que a la fecha de escritura de éste informe no se ha realizado.
	\begin{itemize}
	  \item Integración del código de caso de uso físico en la GUI.
	  \subitem El código que a día de hoy existe del caso de uso físico se reduce a
	  lo expuesto en la sección anterior (un código de comunicación simple
	  master/slave) y por tanto aún no lo he integrado en la GUI. Si Ignacio y
	  Carlos completan este código para que pueda ejecutar una instrucción en el
	  Khepera físico, lo integraré en la GUI.
	\end{itemize}
\end{document}
